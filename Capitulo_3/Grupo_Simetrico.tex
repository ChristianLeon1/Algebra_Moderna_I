\section{El grupo Simétrico $S_n$}
\label{sec:Grupo_Simetrico}

Recordemos que $S_n$ con la composición de funciones es un grupo. Cada elemento de $S_n$ se llama una permutación (la cual es una función biyectiva de $\{1, \dots, n\}$ en sí mismo). En este caso si $\sigma \in S_n$ se denotará:
\begin{equation*}
    \sigma = \begin{pmatrix} 1 & 2 & \cdots & n \\ \sigma(1) & \sigma(2) & \cdots & \sigma(n) \end{pmatrix}
\end{equation*}

\begin{defi}
    Un ciclo de longitud $1 \leq k \leq n$ en $S_n$ es un elemento de $S_n$ tal que existen $\{i_1, \dots, i_k\} \subseteq \{1, \dots, n\}$ con $\sigma(i_1) = i_2, \sigma(i_2) = i_3, \dots, \sigma(i_{k-1}) = i_k, \sigma(i_k) = i_1$ y $\sigma(j) = j \ \forall j \notin \{i_1, \dots, i_k\}$.
\end{defi}

Note que en este caso:
\begin{equation*}
\begin{aligned}
    \sigma^2(i_1) &= \sigma(\sigma(i_1)) = \sigma(i_2) = i_3 \quad \text{y} \quad \sigma^k(i_1) = i_1 \\
    \sigma^3(i_1) &= \sigma(\sigma^2(i_1)) = \sigma(i_3) = i_4 \\
    &\vdots \\
    \sigma^l(i_1) &= i_{1+l} \qquad \text{si } 1 \leq l+1 \leq k, \ l \leq k-1, \ \sigma^k(i_j) = i_{j+k-k} = i_j
\end{aligned}
\end{equation*}

Más aún:
\begin{equation*}
    \sigma^l(i_j) = i_{j+l} \qquad 1 \leq j+l \leq k
\end{equation*}
\begin{equation*}
    \sigma^l(i_j) = i_{j+l-k} \qquad j+l > k \quad (1 \leq l \leq k-j)
\end{equation*}

En este caso, en lugar de escribir
\begin{equation*}
    \sigma = \begin{pmatrix} 1 & \cdots & i_1 & i_2 & \cdots & i_k & \cdots & n \\ 1 & \cdots & i_2 & i_3 & \cdots & i_1 & \cdots & n \end{pmatrix}
\end{equation*}
Escribiremos $\sigma = (i_1 \ i_2 \ i_3 \ \dots \ i_k)$ y se tiene que $\sigma$ se puede denotar de $k$ diferentes formas, a saber:
\begin{equation*}
    \sigma = (i_1, \dots, i_k) = (i_2, i_3, \dots, i_k, i_1) = \dots = (i_k, i_1, i_2, \dots, i_{k-1})
\end{equation*}

Si $k=1$, $\sigma(i) = i \ \forall i$, es decir, $\sigma = id$ y se denota por $\sigma = (1) = (2) = \dots = (k) = (n)$.

Si $k=2$, $\sigma(i_1, i_2)$ y se llama una transposición. En este caso:
\begin{equation*}
\begin{aligned}
    \sigma^2(i_1) &= \sigma(i_2) = i_1 \\
    \sigma^2(i_2) &= \sigma(i_1) = i_2
\end{aligned}
\end{equation*}
Así, $\sigma^2 = id$, es decir, $\sigma = \sigma^{-1}$.

\vspace{0.5cm}

Si $\sigma$ es un ciclo de longitud $k$ también se le llama un $k$-ciclo. Observe que si $\sigma$ es un $k$-ciclo, $\sigma^k = id$, de hecho $|\sigma| = k$.

\begin{equation*}
    \sigma^l(i_j) = \begin{cases} 
        i_{j+l} & \text{si } 1 \leq l \leq k-j \\
        i_{l-k+j} & \text{si } k-j < l \leq k
    \end{cases}
\end{equation*}

En particular si $k=l$, $\sigma^k(i_j) = i_{k-k+j} = i_j$.
Así $|\sigma| \mid k$. Además $\sigma^l(i_1) \neq i_1, \forall 1 \leq l < k$ así que $|\sigma| \geq k$, de donde $|\sigma| = k$.

\begin{defi}
    Diremos que dos ciclos $\sigma, \tau \in S_n$ son disjuntos si:
    \begin{enumerate}
        \item[i)] Cuando $\sigma(i_1) = i_2$ con $i_1 \neq i_2$ se tiene que $\tau(i_1) = i_1$.
        \item[ii)] Cuando $\tau(i_1) = i_2$ con $i_1 \neq i_2$ se tiene que $\sigma(i_1) = i_1$.
    \end{enumerate}
    (Parafraseando, los elementos que mueve $\sigma$, $\tau$ los fija y recíprocamente).
\end{defi}

\begin{eje}
    Consideremos $S_5$, entonces $\sigma = (1 \ 2)$ y $\tau = (3 \ 4 \ 5)$ son disjuntos.
\end{eje}

\NotaBox{}{
    Si $\sigma = (i_1, \dots, i_k)$ y $\tau = (j_1, \dots, j_l)$, entonces $\sigma$ y $\tau$ son disjuntos si y sólo si
    \begin{equation*}
        \{i_1, \dots, i_k\} \cap \{j_1, \dots, j_l\} = \emptyset
    \end{equation*}
}

\ObservacionBox{}{
    Si $\sigma, \tau$ son ciclos disjuntos, entonces $\sigma \circ \tau = \tau \circ \sigma$, es decir, conmutan.
}

\begin{proof}
    Sea $\sigma = (i_1, \dots, i_k)$ y $\tau = (j_1, \dots, j_l)$. Claramente $\sigma \circ \tau$ y $\tau \circ \sigma$ tienen el mismo dominio. Ahora si $s \notin \{i_1, \dots, i_k, j_1, \dots, j_l\}$:
    \begin{equation*}
    \begin{aligned}
        (\sigma \circ \tau)(s) &= \sigma(\tau(s)) = \sigma(s) = s \\
        (\tau \circ \sigma)(s) &= \tau(\sigma(s)) = \tau(s) = s
    \end{aligned}
    \end{equation*}
    Si $s \in \{i_1, \dots, i_k\}$ entonces $s \notin \{j_1, \dots, j_l\}$, así que:
    \begin{equation*}
        (\sigma \circ \tau)(s) = \sigma(\tau(s)) = \sigma(s)
    \end{equation*}
    Más aún $\sigma(s) \in \{i_1, \dots, i_k\}$, luego $\sigma(s) \notin \{j_1, \dots, j_l\}$ y $\tau(\sigma(s)) = \sigma(s)$, por lo tanto $(\tau \circ \sigma)(s) = (\sigma \circ \tau)(s)$.
    
    Por simetría si $s \in \{j_1, \dots, j_l\}$, $(\tau \circ \sigma)(s) = (\sigma \circ \tau)(s)$ en cualquier caso se tiene la igualdad y por lo tanto $\sigma \circ \tau = \tau \circ \sigma$.
\end{proof}

\TeoremaBox{}{
    Sea $\theta \in S_n$, entonces $\theta$ se puede representar de manera única como producto de ciclos ajenos (disjuntos) salvo orden.
}

\begin{proof}
    Sea $\{i_1, \dots, i_k\}$ los elementos que mueve $\theta$ y procedamos por inducción sobre $k$.
    
    Para $k=1$, $\theta=id$ no hay nada que ver $id=(1)$.
    
    Para $k=2$, $\theta=(i_1, i_2)$ y ya se tiene.
    
    Suponga el resultado para todo $1 \leq l \leq k$ y considere que $\theta$ mueve a los elementos $\{i_1, \dots, i_k, i_{k+1}\}$. Considere que $\theta$ mueve a los elementos $\{i_1, \dots, i_k, i_{k+1}\}$. Observe que $i_1, \theta(i_1), \theta^2(i_1), \dots, \theta^{k+1}(i_1) \in \{i_1, \dots, i_{k+1}\}$, por lo tanto $\{i_1, \theta(i_1), \dots, \theta^{k+1}(i_1)\} \subseteq \{i_1, \dots, i_{k+1}\}$, luego existen $1 \leq l, l' \leq k+2$ y $\theta^l(i_1) = \theta^{l'}(i_1)$ y podemos suponer $l > l'$, en este caso $\theta^{l-l'}(i_1) = i_1$, y $1 \leq l-l' \leq k+2-l' \leq k+1$, es decir existe $p$ tal que $\theta^p(i_1) = i_1$, $1 \leq p \leq k+1$.
    
    Sea $p$ el mínimo entero para el cual pasa esto. \par Sea $\sigma = (i_1, \theta(i_1), \dots, \theta^{p-1}(i_1))$. 
    $\{i_1, \theta(i_1), \dots, \theta^{p-1}(i_1)\} \subseteq \{i_1, \dots, i_{k+1}\}$ y definamos
    \begin{equation*}
        \tau(i_j) = \begin{cases} 
            \sigma(i_j) & \text{si } \sigma(i_j) = \theta^l(i_1) \text{ para algún } 1 \leq l \leq p-1 \\
            \theta(i_j) & \text{si } i_j \neq \theta^l(i_1) \ \forall 1 \leq l \leq p-1
        \end{cases}
    \end{equation*}
    
    Entonces $\sigma \circ \tau = \theta$. Si $j \notin \{i_1, \dots, i_{k+1}\}$ entonces $j \neq \theta^l(i_1) \ \forall 1 \leq l \leq p-1$ y entonces
    \begin{equation*}
        (\sigma \circ \tau)(j) = \sigma(\tau(j)) = \sigma(\theta(j)) = \sigma(j) = j = \theta(j)
    \end{equation*}
    
    Si $j \in \{i_1, \dots, i_{k+1}\} \cap \{i_1, \theta(i_1), \dots, \theta^{p-1}(i_1)\}$, entonces $j = \theta^l(i_1)$ con $0 \leq l \leq p-1$.
    \begin{equation*}
        \theta(j) = \theta^{l+1}(i_1) \quad \text{y} \quad \sigma \circ \tau(j) = \sigma(\tau(j)) = \sigma(\theta^l(i_1))= \theta^{l+1}(i_1) =\theta(j)
    \end{equation*}
    (Definición: Como $j$ está en el soporte de $\sigma$, $\tau(j)=\theta(j)$, luego $\sigma(\theta(j)) = \theta(j)$. Y como $\sigma$ es el ciclo $(i_1, \dots, \theta^{p-1}(i_1))$, $\sigma(j) = \theta(j)$).
    En cualquier caso $\theta(j) = (\sigma \circ \tau)(j)$ por lo tanto $\theta = \sigma \circ \tau$.
    
    Ahora $\tau$ mueve a los elementos $\{i_1, \dots, i_{k+1}\} \setminus \{i_1, \sigma(i_1), \dots, \sigma^{p-1}(i_1)\}$ cuya cardinalidad es menor o igual a $k$. Por hipótesis de inducción $\tau$ es producto de ciclos disjuntos y por lo tanto $\theta$ lo es.
    
    Unicidad: Suponga ahora que
    \begin{equation*}
        \theta = \sigma_1 \dots \sigma_k = \tau_1 \dots \tau_l
    \end{equation*}
    Con los $\sigma_i$'s ciclos disjuntos a pares y los $\tau_i$'s ciclos disjuntos a pares.
    
    Si $\theta = id$ no hay nada que ver, si no, sea $i_1$ un elemento que mueve $\theta$, entonces existen $1 \leq i \leq k$ y $1 \leq j \leq l$ tales que $\sigma_i, \tau_j$ mueven a $i_1$, de hecho $i, j$ son únicos pues los $\sigma_i$'s y los $\tau_i$'s son disjuntos o más aún como son disjuntos conmutan, por lo que podemos pensar que $i=1=j$, en este caso $\sigma_2, \dots, \sigma_k, \tau_2, \dots, \tau_l$ no mueven a $i_1$.
    
    Además $\sigma_1 = (i_1, \sigma_1(i_1), \dots, \sigma_1^{s-1}(i_1))$, $\tau_1 = (i_1, \tau(i_1), \dots, \tau^{r-1}(i_1))$ con $s$ y $r$ los órdenes de $\sigma_1$ y $\tau_1$ respectivamente. Como los $\sigma_i$'s son disjuntos $\sigma_j$ no mueve a ninguno de $\{i_1, \sigma(i_1), \dots, \sigma^{k-1}(i_1)\} \ \forall 1 < j \leq k$. Análogamente $\tau_j$ no mueve a ninguno de \par $\{i_1, \tau(i_1), \dots, \tau^{l-1}(i_1)\} \ \forall 1 < j \leq l$. Por tanto para $m \in \mathbb{N}$
    \begin{equation*}
    \begin{aligned}
        \theta^m(i_1) &= (\sigma_1 \dots \sigma_k)^m(i_1) = (\sigma_1 \dots \sigma_k)^{m-1}((\sigma_1 \dots \sigma_k)(i_1)) \\
        &= (\sigma_1 \dots \sigma_k)^{m-1}(\sigma_1(i_1)) = \dots = \sigma_1^m(i_1)
    \end{aligned}
    \end{equation*}
    
    Análogamente
    \begin{equation*}
    \begin{aligned}
        \theta^m(i_1) &= \tau_1^m(i_1) \\
        \sigma_1^m(i_1) &= \tau_1^m(i_1)
    \end{aligned}
    \end{equation*}
    
    Ahora podemos suponer que $s \leq r$, entonces
    \begin{equation*}
        i_1 = \sigma_1^s(i_1) = \tau_1^s(i_1)
    \end{equation*}
    $s > r-1$ o $s \geq r$, de donde $s=r$.
    
    Más aún, como $\sigma_1^m(i_1) = \tau_1^m(i_1) \ \forall m \in \mathbb{N}$ se tiene que $\sigma_1 = \tau_1$.
    Ahora de la igualdad $\theta = \sigma_1 \dots \sigma_k = \tau_1 \dots \tau_l$ se obtiene $\sigma_2 \dots \sigma_k = \tau_2 \dots \tau_l$.
    Podemos suponer $k \leq l$. En cuyo caso realizando el mismo proceso obtenemos
    $\sigma_k = \tau_k$ y $Id = \tau_{k+1} \dots \tau_l$.
    Como los $\tau_j$ son disjuntos a pares necesariamente $\tau_{k+1} \dots \tau_l$ tienen longitud uno. Por lo tanto $l=k$.
\end{proof}

\CorolarioBox{}{
    Sea $\theta \in S_n$, entonces el orden de $\theta$ es el mínimo común múltiplo de los órdenes de los ciclos que aparecen en su factorización.
}

\begin{proof}
    Sea $\theta = \sigma_1 \dots \sigma_k$ con los $\sigma_i$'s ciclos disjuntos a pares y $n_i = o(\sigma_i)$, 
    $m = [n_1, \dots, n_k]$ entonces
    \begin{equation*}
        \theta^m = (\sigma_1 \dots \sigma_k)^m = \sigma_1^m \dots \sigma_k^m = id
    \end{equation*}
    De donde $o(\theta) \mid m$.
    Si $o(\theta) < m$, $Id = \theta^{o(\theta)} = \sigma_1^{o(\theta)} \dots \sigma_k^{o(\theta)}$, por lo cual $n_i \mid o(\theta) \ \forall i$, de donde $m = [n_1, \dots, n_k] \mid o(\theta)$ de donde $m = o(\theta)$.
\end{proof}

\ObservacionBox{}{
    \begin{enumerate}
        \item Sea $(i \ j)$ una transposición de $S_n$, entonces $(i \ j) = (1 \ i)(1 \ j)(1 \ i)$.
        \item Sean $x_1, \dots, x_n$ variables. Definimos
        \begin{equation*}
            P_n = P_n(x_1, \dots, x_n) = \prod_{1 \leq i < j \leq n} (x_i - x_j) \quad \forall n \geq 2
        \end{equation*}
        y hagamos actuar el grupo de permutaciones sobre $P_n$ de la siguiente forma. Para $\theta \in S_n$:
        \begin{equation*}
            \theta(P_n) = P_n^\theta = \prod_{1 \leq i < j \leq n} (x_{\theta(i)} - x_{\theta(j)})
        \end{equation*}
        \item Si $1 < k \leq n$ y $\theta = (1, k)$, entonces $\theta(P_n) = -P_n$.
        \item Si $\theta = (i \ j)$ es una transposición, $\theta(P_n) = -P_n$.
        \item Sea $\theta \in S_n$ un $r$-ciclo, entonces $\theta$ es producto de transposiciones (se puede representar como un producto de transposiciones, no necesariamente única).
    \end{enumerate}
}

\begin{proof}[Demostración de 1)]
    Queremos demostrar que la transposición $(i \ j)$ es igual al producto $(1 \ i)(1 \ j)(1 \ i)$.
    Llamemos $\sigma = (1 \ i)(1 \ j)(1 \ i)$ y evaluemos su acción sobre los elementos de $\{1, \dots, n\}$ actuando de derecha a izquierda (orden de composición de funciones). Asumimos $1, i, j$ distintos (si alguno es igual, la igualdad es trivial).

    \begin{itemize}
        \item Para el elemento $1$:
        \begin{equation*}
            1 \xrightarrow{(1 \ i)} i \xrightarrow{(1 \ j)} i \xrightarrow{(1 \ i)} 1
        \end{equation*}
        Luego $\sigma(1) = 1$. (El $1$ queda fijo, lo cual es correcto pues $(i \ j)$ no mueve al $1$).

        \item Para el elemento $i$:
        \begin{equation*}
            i \xrightarrow{(1 \ i)} 1 \xrightarrow{(1 \ j)} j \xrightarrow{(1 \ i)} j
        \end{equation*}
        Luego $\sigma(i) = j$.

        \item Para el elemento $j$:
        \begin{equation*}
            j \xrightarrow{(1 \ i)} j \xrightarrow{(1 \ j)} 1 \xrightarrow{(1 \ i)} i
        \end{equation*}
        Luego $\sigma(j) = i$.

        \item Para cualquier otro elemento $k \notin \{1, i, j\}$:
        Todas las transposiciones en el producto fijan a $k$, por lo tanto $\sigma(k) = k$.
    \end{itemize}

    Como $\sigma$ intercambia $i$ con $j$ y deja fijos a los demás elementos (incluido el $1$), concluimos que $\sigma = (i \ j)$.
\end{proof}

\begin{proof}[Demostración de 2)]
    Queremos verificar que la acción definida cumple con la propiedad de grupo $(\tau \circ \theta)(P_n) = \tau(\theta(P_n))$.
    
    Recordemos que la acción de una permutación $\sigma$ sobre un polinomio en variables $x_1, \dots, x_n$ consiste en reemplazar cada subíndice $k$ por $\sigma(k)$. Es decir, $\sigma$ actúa sobre las posiciones de las variables.
    
    Sea $P_n = \prod (x_i - x_j)$.
    Primero aplicamos $\theta$:
    \begin{equation*}
        \theta(P_n) = \prod_{1 \leq i < j \leq n} (x_{\theta(i)} - x_{\theta(j)})
    \end{equation*}
    Llamemos $Q(x_1, \dots, x_n)$ a este nuevo polinomio. Notemos que el término que ocupaba la posición asociada al índice $k$ ahora tiene el índice $\theta(k)$.
    
    Ahora aplicamos $\tau$ al polinomio $Q$. La regla dice que $\tau$ reemplaza cualquier variable con índice $m$ por la variable con índice $\tau(m)$.
    En $Q$, tenemos variables de la forma $x_{\theta(i)}$. Aquí el índice es $m = \theta(i)$.
    Al aplicar $\tau$, reemplazamos $x_{\theta(i)}$ por $x_{\tau(\theta(i))}$.
    \begin{equation*}
        \tau(\theta(P_n)) = \prod_{1 \leq i < j \leq n} (x_{\tau(\theta(i))} - x_{\tau(\theta(j))})
    \end{equation*}
    
    Por otro lado, consideremos la permutación compuesta $\rho = \tau \circ \theta$.
    Si aplicamos $\rho$ directamente a $P_n$, reemplazamos cada índice $k$ por $\rho(k) = \tau(\theta(k))$.
    \begin{equation*}
        (\tau \circ \theta)(P_n) = \prod_{1 \leq i < j \leq n} (x_{(\tau \circ \theta)(i)} - x_{(\tau \circ \theta)(j)}) = \prod_{1 \leq i < j \leq n} (x_{\tau(\theta(i))} - x_{\tau(\theta(j))})
    \end{equation*}
    
    Comparando ambas expresiones, tenemos que $(\tau \circ \theta)(P_n) = \tau(\theta(P_n))$.
\end{proof}

\begin{proof}[Demostración de 3)]
    Procedamos por inducción sobre $n$.
    
    Para $n=2$, tenemos $1 < k \leq 2 \implies k=2$, luego $\theta=(1, 2)$.
    \begin{equation*}
        \theta(P_2) = \theta(x_1 - x_2) = x_{\theta(1)} - x_{\theta(2)} = x_2 - x_1 = -(x_1 - x_2) = -P_2
    \end{equation*}
    
    Suponga el resultado cierto para $n$, es decir, para $1 < k \leq n$. Note que:
    \begin{equation*}
    \begin{aligned}
        P_{n+1} &= \prod_{1 \leq i < j \leq n+1} (x_i - x_j) = \prod_{1 \leq i < j \leq n} (x_i - x_j) \cdot \prod_{1 \leq i \leq n} (x_i - x_{n+1}) \\
        &= P_n \cdot \prod_{1 \leq i \leq n} (x_i - x_{n+1})
    \end{aligned}
    \end{equation*}
    
    Caso A: Si $1 < k \leq n$, entonces $\theta = (1, k) \in S_n \subseteq S_{n+1}$ (fija a $n+1$).
    \begin{equation*}
    \begin{aligned}
        \theta(P_{n+1}) &= \theta \left( P_n \cdot \prod_{1 \leq i \leq n} (x_i - x_{n+1}) \right) \\
        &= \theta(P_n) \cdot \theta \left( \prod_{1 \leq i \leq n} (x_i - x_{n+1}) \right) \\
        &\overset{\text{H.I.}}{=} (-P_n) \cdot \prod_{1 \leq i \leq n} (x_{\theta(i)} - x_{n+1})
    \end{aligned}
    \end{equation*}
    Como $\theta$ solo permuta los elementos $\{1, \dots, n\}$, el conjunto de factores en la productoria es el mismo (solo cambia el orden), por lo tanto el producto permanece invariante.
    \begin{equation*}
        = -P_n \cdot \prod_{1 \leq i \leq n} (x_i - x_{n+1}) = -P_{n+1}
    \end{equation*}
    
    Caso B: Si $k=n+1$, entonces $\theta=(1, n+1)$. Descomponemos $P_{n+1}$ cuidadosamente:
    \begin{align*}
        \theta(P_{n+1}) &= \theta \left[ \prod_{1 < i < j \leq n} (x_i - x_j) \cdot \prod_{1 < j \leq n} (x_1 - x_j) \cdot \prod_{1 < i < n+1} (x_i - x_{n+1}) \cdot (x_1 - x_{n+1}) \right] \\
        &= \prod_{1 < i < j \leq n} (x_i - x_j) \cdot \prod_{1 < j \leq n} (x_{n+1} - x_j) \cdot \prod_{1 < i < n+1} (x_i - x_1) \cdot (x_{n+1} - x_1) \\
        &= \prod_{1 < i < j \leq n} (x_i - x_j) \cdot \prod_{1 < j \leq n} [-(x_j - x_{n+1})] \cdot \prod_{1 < i < n+1} [-(x_1 - x_i)] \cdot [-(x_1 - x_{n+1})] \\
        &- \left[ \prod_{1 < i < j \leq n} (x_i - x_j) \cdot \prod_{1 < j \leq n} (x_j - x_{n+1}) \cdot \prod_{1 < i < n+1} (x_1 - x_i) \cdot (x_1 - x_{n+1}) \right] \\
        &= - \left( \prod_{1 \leq i < j \leq n} (x_i - x_j) \cdot \prod_{1 \leq j \leq n} (x_j - x_{n+1}) \right) \\
        &=P_n \cdot \prod_{1 \leq i \leq n} (x_i - x_{n+1}) = - P_{n+1}
    \end{align*}
\end{proof}

\begin{proof}[Demostración de 4)]
    Si $\theta = (i \ j)$ es una transposición, por la observación 1 sabemos que $(i \ j) = (1 \ i)(1 \ j)(1 \ i)$. Usando la propiedad 3 repetidamente:
    \begin{equation*}
    \begin{aligned}
        \theta(P_n) &= [(1 \ j)(1 \ i)(1 \ j)] P_n \\
        &= (1 \ j)((1 \ i)((1 \ j) P_n)) \\
        &= (1 \ j)((1 \ i)(-P_n)) \\
        &= (1 \ j)(-(-P_n)) \\
        &= (1 \ j)(P_n) = -P_n
    \end{aligned}
    \end{equation*}
\end{proof}

\begin{proof}[Demostración de 5)]
    Sea $\theta = (i_1, \dots, i_k)$ un $k$-ciclo. Podemos descomponerlo verificando la acción sobre los elementos:
    \begin{equation*}
        (i_1, \dots, i_k) = (i_1, i_2)(i_2, i_3) \dots (i_{k-2}, i_{k-1})(i_{k-1}, i_k)
    \end{equation*}
\end{proof}

\begin{eje}
    $(1 \ 2 \ 3) = (1 \ 2)(2 \ 3)$.
    También se puede escribir de forma no única, por ejemplo:
    $(1 \ 2 \ 3) = (4 \ 5)(1 \ 2)(2 \ 3)(4 \ 5) = (1 \ 2)(1 \ 2)(1 \ 2)(2 \ 3)(3 \ 2)(3 \ 2)$.
\end{eje}