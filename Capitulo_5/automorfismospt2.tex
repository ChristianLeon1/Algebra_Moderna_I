\TeoremaBox{}{
    Sea $G$ un grupo cíclico de orden $n$. Entonces:
    \begin{equation*}
        \text{Aut}(G) \cong (\mathbb{Z}/n\mathbb{Z})^*
    \end{equation*}
    Donde $(\mathbb{Z}/n\mathbb{Z})^*$ es el grupo multiplicativo de las unidades módulo $n$.
    En particular, $|\text{Aut}(G)| = \varphi(n)$, donde $\varphi$ es la función de Euler.
}

\begin{proof}
    Sea $G = \langle g \rangle$ un grupo cíclico de orden $n$.
    Sea $f \in \text{Aut}(G)$. Como $G$ es cíclico, $f$ queda completamente determinado por la imagen del generador $g$.
    Sea $f(g) = g^{c_f}$ para algún entero $c_f$.
    Como $f$ es un automorfismo, $f(g)$ debe ser otro generador de $G$.
    Sabemos que $g^k$ es un generador de $G$ si y solo si $\gcd(k, n) = 1$.
    Por lo tanto, $\gcd(c_f, n) = 1$, lo que implica que la clase $\overline{c_f}$ pertenece a $(\mathbb{Z}/n\mathbb{Z})^*$.
    
    Definimos la función $\Psi: \text{Aut}(G) \to (\mathbb{Z}/n\mathbb{Z})^*$ dada por:
    \begin{equation*}
        \Psi(f) = \overline{c_f} \quad \text{donde } f(g) = g^{c_f}
    \end{equation*}
    
    \textbf{1. $\Psi$ es un homomorfismo:}
    Sean $f_1, f_2 \in \text{Aut}(G)$ con $f_1(g) = g^{c_{f_1}}$ y $f_2(g) = g^{c_{f_2}}$.
    \begin{equation*}
        (f_1 \circ f_2)(g) = f_1(f_2(g)) = f_1(g^{c_{f_2}}) = (f_1(g))^{c_{f_2}} = (g^{c_{f_1}})^{c_{f_2}} = g^{c_{f_1} c_{f_2}}
    \end{equation*}
    Por lo tanto, el exponente asociado a la composición es el producto de los exponentes:
    \begin{equation*}
        \Psi(f_1 \circ f_2) = \overline{c_{f_1} c_{f_2}} = \overline{c_{f_1}} \cdot \overline{c_{f_2}} = \Psi(f_1) \Psi(f_2)
    \end{equation*}
    
    \textbf{2. $\Psi$ es inyectiva:}
    \begin{equation*}
        \Psi(f) = \overline{1} \implies c_f \equiv 1 \pmod n \implies f(g) = g^1 = g
    \end{equation*}
    Como el automorfismo fija al generador, fija a todo el grupo. Luego $f = Id_G$.
    
    \textbf{3. $\Psi$ es sobreyectiva:}
    Sea $\overline{k} \in (\mathbb{Z}/n\mathbb{Z})^*$. Entonces $\gcd(k, n) = 1$.
    Definimos $f: G \to G$ por $f(x) = x^k$.
    Como $\gcd(k, n) = 1$, la aplicación $x \mapsto x^k$ es una biyección en el grupo cíclico finito y es un homomorfismo ($f(xy) = (xy)^k = x^k y^k$ pues $G$ es abeliano).
    Así, $f \in \text{Aut}(G)$ y $\Psi(f) = \overline{k}$.
    
    Por lo tanto, $\text{Aut}(G) \cong (\mathbb{Z}/n\mathbb{Z})^*$.
\end{proof}

\begin{cor}
    Si $p$ es un número primo, entonces:
    \begin{equation*}
        \text{Aut}(\mathbb{Z}_p) \cong \mathbb{Z}_{p-1}
    \end{equation*}
\end{cor}

\begin{proof}
    Aplicando el teorema anterior con $G = \mathbb{Z}_p$ (cíclico de orden $p$), tenemos:
    \begin{equation*}
        \text{Aut}(\mathbb{Z}_p) \cong (\mathbb{Z}/p\mathbb{Z})^*
    \end{equation*}
    El orden de este grupo es $\varphi(p) = p-1$.
    Sabemos por la teoría de grupos finitos (específicamente por la existencia de raíces primitivas módulo $p$) que el grupo multiplicativo de un cuerpo finito $\mathbb{Z}_p$ es siempre cíclico.
    Por lo tanto:
    \begin{equation*}
        (\mathbb{Z}/p\mathbb{Z})^* \cong C_{p-1} \cong \mathbb{Z}_{p-1}
    \end{equation*}
    Así concluimos que $\text{Aut}(\mathbb{Z}_p)$ es isomorfo al grupo cíclico de orden $p-1$.
\end{proof}

\begin{eje}
    Sea $G$ un grupo y hagamos actuar $G$ en sí mismo por conjugación.
    Entonces esta es una acción como grupo.
    
    Definimos $\tilde{\varphi}: G \times G \to G$ por $\tilde{\varphi}(g, g_1) = g g_1 g^{-1}$.
    Ya se tiene que es una acción (visto en el capítulo anterior). Veamos que actúa por automorfismos:
    \begin{equation*}
        (k_1 k_2)^g = \tilde{\varphi}(g, k_1 k_2) = g (k_1 k_2) g^{-1} = g k_1 (g^{-1} g) k_2 g^{-1}
    \end{equation*}
    \begin{equation*}
        = (g k_1 g^{-1}) (g k_2 g^{-1}) = \tilde{\varphi}(g, k_1) \tilde{\varphi}(g, k_2) = k_1^g k_2^g
    \end{equation*}
    Por lo tanto, la conjugación preserva el producto.
\end{eje}

\begin{obs}
    Sean $H$ y $K$ grupos. Entonces $H$ actúa en $K$ como grupo si y solo si $\varphi: H \to \text{Aut}(K)$ es un homomorfismo.
    (Nota: $\text{Aut}(K) \le S_K$).
\end{obs}

\begin{enefecto}
    $\Rightarrow)$ Si $H$ actúa como grupo en $K$, existe $\tilde{\varphi}: H \times K \to K$ que cumple las condiciones i), ii) de acción y la condición de compatibilidad iii).
    Definamos $\varphi: H \to \text{Aut}(K)$ por $\varphi(h) = f_h$, donde $f_h: K \to K$ está dada por $f_h(k) = \tilde{\varphi}(h, k)$.
    
    \textbf{1. $\varphi$ está bien definida:}
    Como $\tilde{\varphi}$ es acción, la función $\varphi: H \to S_K$ dada por $h \mapsto f_h$ es un homomorfismo de grupos (propiedad general de acciones).
    Basta ver que la imagen cae en $\text{Aut}(K)$, es decir, que $f_h \in \text{Aut}(K)$ para todo $h$.
    Sabemos que $f_h$ es biyectiva (por ser acción). Además:
    \begin{equation*}
        f_h(k_1 k_2) = \tilde{\varphi}(h, k_1 k_2) = (k_1 k_2)^h = k_1^h k_2^h
    \end{equation*}
    \begin{equation*}
        = \tilde{\varphi}(h, k_1) \tilde{\varphi}(h, k_2) = f_h(k_1) f_h(k_2)
    \end{equation*}
    Luego, $f_h$ es homomorfismo y por tanto $f_h \in \text{Aut}(K)$. Así $\varphi$ está bien definida.
    
    $\Leftarrow)$ Recíprocamente, sea $\varphi: H \to \text{Aut}(K)$ un homomorfismo.
    Definamos $\tilde{\varphi}: H \times K \to K$ por $\tilde{\varphi}(h, k) = (\varphi(h))(k)$.
    Notemos que como $\varphi(h) \in \text{Aut}(K)$, denotemos $f_h = \varphi(h)$, entonces $\tilde{\varphi}(h, k) = f_h(k)$.
    
    Verifiquemos que es acción como grupo:
    \begin{itemize}
        \item $k^e = \tilde{\varphi}(e, k) = (\varphi(e))(k) = Id_K(k) = k$. (Pues $\varphi$ es homomorfismo, $\varphi(e) = Id$).
        \item $(k^{h_1})^{h_2} = \tilde{\varphi}(h_2, k^{h_1}) = f_{h_2}(f_{h_1}(k)) = (f_{h_2} \circ f_{h_1})(k)$.
        Como $\varphi$ es homomorfismo, $f_{h_2} \circ f_{h_1} = \varphi(h_2) \circ \varphi(h_1) = \varphi(h_2 h_1) = f_{h_2 h_1}$.
        Luego, $= f_{h_2 h_1}(k) = k^{h_2 h_1}$.
        \item 
        $(k_1 k_2)^h = f_h(k_1 k_2) = f_h(k_1) f_h(k_2)$ (pues $f_h \in \text{Aut}(K)$).
        $= k_1^h k_2^h$.
    \end{itemize}
\end{enefecto}

\begin{eje}
    Sea $G$ un grupo y $K \trianglelefteq G$. Entonces $G$ \textbf{no} actúa necesariamente en $K$ como grupo por multiplicación a la izquierda.
    Es decir, sea $\tilde{\varphi}(g, k) = gk$.
    
    Aunque $\tilde{\varphi}(e, k) = ek = k$ y $\tilde{\varphi}(g_1 g_2, k) = (g_1 g_2)k = g_1(g_2 k)$, la propiedad de automorfismo falla:
    \begin{equation*}
        (k_1 k_2)^g = g(k_1 k_2)
    \end{equation*}
    mientras que:
    \begin{equation*}
        k_1^g k_2^g = (g k_1)(g k_2) = g k_1 g k_2
    \end{equation*}
    En general $g k_1 k_2 \neq g k_1 g k_2$ (esto implicaría $e = g$, lo cual no es cierto para todo $g$).
    Además, $gk$ no necesariamente está en $K$ si $K$ no es el grupo total (aunque si tomamos $K=G$, falla la condición de homomorfismo a menos que $G$ sea trivial).
\end{eje}