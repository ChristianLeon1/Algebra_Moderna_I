\section{Cardinalidad de conjuntos}%
\label{sec:Cardinalidad de conjuntos}

Dado un conjunto $A$, se denotará su cardinalidad (número de elementos) como $|A|$. Si $A$ es un conjunto finito, entonces $|A|$ es un número natural. Si $A$ es infinito, entonces $|A|=\infty$.

\ObservacionBox{}{
	Sean $A, B$, conjuntos finitos, con $B \subseteq A$. Entonces:
	\begin{equation*}
		|A \setminus B| = |A| - |B|
	\end{equation*}
}

\begin{enefecto}
	basta notar que $B \cup (A \setminus B) = A$ y que $B \cap (A \setminus B) = \emptyset$, luego $|A| = |B \cup (A \setminus B)| = |B| + |A \setminus B|$, así $|A \setminus B = |A| - |B|$.
\end{enefecto}

\ObservacionBox{}{
	Sean $A$ y $B$ dos conjuntos finitos, entonces:
	\begin{equation*}
		|A \cup B| = |A| + |B| - |A \cap B|
	\end{equation*}
}

\begin{enefecto}
	Sean $A$ y $B$ conjuntos finitos, note que:
	\begin{equation*}
		A \cup B = (A \setminus (A \cap B)) \cup (B \setminus (A \cap B)) \cup (A \cap B)
	\end{equation*}

	Además: $(A \setminus (A \cap B))$, $(B \setminus (A \cap B))$, $(A \cap B)$, son disjuntos, más aún:
	\begin{align*}
		|A \setminus (A \cap B)| & = |A| - |A \cap B| \\
		|B \setminus (A \cap B)| & = |B| - |A \cap B|
	\end{align*}

	Así:
	\begin{align*}
		|A \cup B| & = |A \setminus (A \cap B)| + |B \setminus (A \cap B)| + |A \cap B| \\
		           & = |A| - |A \cap B| + |B| - |A \cap B| + |A \cap B|                 \\
		           & = |A| + |B| - |A \cap B|
	\end{align*}
\end{enefecto}

\begin{prop}[Principio de inclusión exclusión]
	\label{prop: Principio de inclusión exclusión}
	Sean $A_1, \dots, A_n$ conjuntos finitos, se tiene:
	\begin{equation*}
		\left| \bigcup_{i=1}^n A_i \right| = \sum_{i=1}^n |A_i| - \sum_{1 \le i_1 < i_2 \le n} |A_{i_1} \cap A_{i_2}| + \dots + (-1)^{n-1} |A_1 \cap \dots \cap A_n|
	\end{equation*}

\end{prop}

\ObservacionBox{}{
	Suponga que $C_1$ es la condición que cumplen los elementos $A$ y $C_2$ los de $B$, i.e.:
	\begin{align*}
		A & = \{x \in \Omega : x \text{ cumple } C_1\} \\
		B & = \{x \in \Omega : x \text{ cumple } C_2\}
	\end{align*}

	Denotemos $N(C_i)$ a la cantidad de elementos que cumplen $C_i$, $N(C_1, C_2)$ a los que cumplen ambas, $N(\bar{C}_i)$ a los que no cumplen y $N(\bar{C}_1, \bar{C}_2)$ los que no cumplen $C_1$ ni $C_2$, entonces:
	\begin{equation*}
		N(\bar{C}_1, \bar{C}_2) = |\Omega| - \left( N(C_1) + N(C_2) - N(C_1, C_2) \right)
	\end{equation*}
}

\begin{enefecto}
	Note que:
	\begin{align*}
		N(\bar{C}_1, \bar{C}_2) & = |A^c \cap B^c| = |(A \cup B)^c| = |\Omega \setminus (A \cup B)| = |\Omega| - |A \cup B| \\
		                        & = |\Omega| - (|A| + |B| - |A \cap B|)                                                     \\
		                        & = |\Omega| - (N(C_1) + N(C_2) - N(C_1, C_2))
	\end{align*}
\end{enefecto}

\begin{eje}
	Sea $\Omega = \{x \in \mathbb{Z} : 1 \le x \le 1000\}$ ¿Cuántos enteros de estos no son divisibles por 3 o 5?
\end{eje}

\textit{ Sol. } Consideremos:
\begin{align*}
	C_1 & : x \text{ sea divisible por } 3 \\
	C_2 & : x \text{ sea divisible por } 5
\end{align*}

Así $N(C_1) = 333$, $N(C_2) = 200$, $N(C_1, C_2) = 66$.

Luego:
\begin{align*}
	N(\bar{C}_1, \bar{C}_2) & = |\Omega| - (N(C_1) + N(C_2) - N(C_1, C_2)) \\
	                        & = 1000 - (333 + 200 - 66)                    \\
	                        & = 533
\end{align*}

Sea $A_1, \dots, A_n$ una colección finita de conjuntos finitos, definidos:
\begin{equation*}
	A_i = \{ x \in \Omega : x \text{ cumpla } C_i \}, \quad C_i \text{ condición.}
\end{equation*}

Definamos de este modo:
\begin{align*}
	S_1 & = N(C_1) + \dots + N(C_n)                                                   \\
	S_2 & = N(C_1, C_2) + \dots + N(C_1, C_n) + N(C_2, C_3) + \dots + N(C_{n-1}, C_n) \\
	    & \vdots                                                                      \\
	S_i & = \sum_{1 \le j_1 < \dots < j_i \le n} N(C_{j_1}, \dots, C_{j_i})           \\
	    & \vdots                                                                      \\
	S_n & = N(C_1, \dots, C_n)
\end{align*}

Por el principio de inclusión exclusión generalizado:
\begin{equation*}
	N(\bar{C}_1, \dots, \bar{C}_n) = |\Omega| - \left( S_1 - S_2 + \dots + (-1)^{n-1} S_n \right)
\end{equation*}
