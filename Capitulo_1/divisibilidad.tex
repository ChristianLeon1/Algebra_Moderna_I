\section{Divisibilidad}%
\label{sec:Divisibilidad}

\begin{defi}[Divisibilidad]
	Sean $a, b \in \mathbb{Z}$, con $a \neq 0$, se dice que $a | b$ si $\exists k \in \mathbb{Z}$ tal que $b = ak$.
\end{defi}

\begin{defi}[Máximo Común Divisor]
	Sea $a, b \in \mathbb{Z}$, al menos uno distinto de cero, definimos a $d \in \mathbb{Z}$ un máximo común divisor de $a$ y $b$, denotado por $(a, b)$, si cumple:
	\begin{enumerate}[label=\roman*), font=\normalfont]
		\item $d > 0$.
		\item $d | a$ y $d | b$.
		\item Si $c | a$ y $c | b$, entonces $c | d$.
	\end{enumerate}
\end{defi}

\begin{prop}[Propiedades de la Divisibilidad]
	Sean $a, b, c \in \mathbb{Z}$, con $a, b \neq 0$, entonces:
	\begin{enumerate}[label=\roman*), font=\normalfont]
		\item Si $a | b$ y $b | c$, entonces $a | c$.
		\item Si $a | b$ y $a | c$, entonces $a | (b + c)$.
		\item Si $a | b$, entonces $a | bk$ para todo $k \in \mathbb{Z}$.
		\item Si $a | b$ y $b \neq 0$, entonces $|a| \leq |b|$.
		\item Si $a | b$ y $b | a$, entonces $a = \pm b$.
		\item Si $a | b$, entonces $(a, b) = |a|$.
		\item Si $c | a$ y $c | b$, entonces $c = ax + by$ para algunos $x, y \in \mathbb{Z}$.
	\end{enumerate}

\end{prop}

\begin{prop}
	Sea $a, b \in \mathbb{Z}$, al menos uno distinto de cero, entonces existe un único máximo común divisor de $a$ y $b$.
\end{prop}

\TeoremaBox{Algoritmo de la división}{
	Sean $a, b \in \mathbb{Z}$, con $b > 0$, entonces existen únicos $q, r \in \mathbb{Z}$ tales que:
	\begin{equation*}
		a = bq + r, \quad 0 \leq r < |b|.
	\end{equation*}
}
