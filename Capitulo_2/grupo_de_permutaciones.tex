\section{Grupo de permutaciones}%
\label{sec:Grupo de permutaciones}

\begin{defi}[Grupo de Permutaciones]
	Sea $X$ un conjunto no vacío, sea $\mathcal{H} = \{ f: X \to X : f \text{ es biyectiva} \}$, consideremos la composición de funciones, entonces $\mathcal{H}$ forma un grupo llamado el grupo de permutaciones del conjunto $X$ denotado por $S_X$.

	En caso de que $X$ sea finito, podemos enlistar los elementos de $X$ por $a_1, \dots, a_n$, podemos representar con un arreglo bidimensional de renglones colocando:
	\begin{equation*}
		\begin{pmatrix}
			a_1           & a_2           & \dots & a_n           \\
			a_{\sigma(1)} & a_{\sigma(2)} & \dots & a_{\sigma(n)}
		\end{pmatrix}
	\end{equation*}
	Donde $\sigma: \{1, \dots, n\} \to \{1, \dots, n\}$, tal que $\sigma(i)=j$, si $f(a_i) = a_j$, de este modo podemos prescindir de los elementos de $X$ y fijarnos solo en los subíndices e identificar a $f$ con:
	\begin{equation*}
		\begin{pmatrix}
			1         & 2         & \dots & n         \\
			\sigma(1) & \sigma(2) & \dots & \sigma(n)
		\end{pmatrix}
	\end{equation*}
	En este caso se escribirá como $S_n$ con $n=|X|$.
\end{defi}

\begin{eje}
	$S_3$ es el grupo formado por los elementos:
	\begin{equation*}
		\{ e, \sigma, \theta, \sigma \cdot \theta, \theta \cdot \sigma, \theta^2 \}
	\end{equation*}
	Donde:
	\begin{align*}
		e                   & = \begin{pmatrix} 1 & 2 & 3 \\ 1 & 2 & 3 \end{pmatrix} &
		\theta              & = \begin{pmatrix} 1 & 2 & 3 \\ 2 & 3 & 1 \end{pmatrix} &
		\sigma              & = \begin{pmatrix} 1 & 2 & 3 \\ 2 & 1 & 3 \end{pmatrix}   \\
		\theta^2            & = \begin{pmatrix} 1 & 2 & 3 \\ 3 & 1 & 2 \end{pmatrix} &
		\sigma \cdot \theta & = \begin{pmatrix} 1 & 2 & 3 \\ 1 & 3 & 2 \end{pmatrix} &
		\theta \cdot \sigma & = \begin{pmatrix} 1 & 2 & 3 \\ 3 & 2 & 1 \end{pmatrix}
	\end{align*}

	\begin{equation*}
		\renewcommand{\arraystretch}{1.5}
		\begin{array}{c|c|c|c|c|c|c}
			\circ               & e                   & \theta              & \sigma              & \theta^2            & \sigma \cdot \theta & \theta \cdot \sigma \\ \hline
			e                   & e                   & \theta              & \sigma              & \theta^2            & \sigma \cdot \theta & \theta \cdot \sigma \\
			\theta              & \theta              & \theta^2            & \theta \cdot \sigma & e                   & \sigma              & \sigma \cdot \theta \\
			\sigma              & \sigma              & \sigma \cdot \theta & e                   & \theta \cdot \sigma & \theta              & \theta^2            \\
			\theta^2            & \theta^2            & e                   & \sigma \cdot \theta & \theta              & \theta \cdot \sigma & \sigma              \\
			\sigma \cdot \theta & \sigma \cdot \theta & \theta \cdot \sigma & \theta^2            & \sigma              & e                   & \theta              \\
			\theta \cdot \sigma & \theta \cdot \sigma & \sigma              & \theta              & \sigma \cdot \theta & \theta^2            & e
		\end{array}
	\end{equation*}

	Es evidente que $S_3$ no es abeliano, basta notar $\theta \circ \sigma \neq \sigma \circ \theta$. Además observe que si el orden de $X$ es $n$, $|S_n| = n!$.
\end{eje}

\ObservacionBox{}{
	Si $n \ge 3$, entonces $S_n$ no es abeliano.
}

\begin{enefecto} Basta tomar:
	\begin{equation*}
		\sigma = \begin{pmatrix}
			1 & \dots & i   & i+1 & \dots & n \\
			1 & \dots & i+1 & i   & \dots & n
		\end{pmatrix}, \quad
		\theta = \begin{pmatrix}
			1 & \dots & j & j+1 & \dots & n \\
			1 & \dots & j & j+1 & \dots & n
		\end{pmatrix} \quad \text{con } i \neq j
	\end{equation*}
	y notar que $\sigma \circ \theta \neq \theta \circ \sigma$.

	Además podemos notar que trivialmente $S_1$ y $S_2$ son un grupo abeliano.
\end{enefecto}
