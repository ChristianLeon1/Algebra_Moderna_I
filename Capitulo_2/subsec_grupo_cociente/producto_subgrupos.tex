\begin{defi}[Producto de subgrupos]
	Sea $(G, \circ, e)$ un grupo, sean $S, T \le G$, definimos el producto de $S$ y $T$ como:
	\begin{equation*}
		ST = \{ st : s \in S, t \in T \}
	\end{equation*}
\end{defi}

\ObservacionBox{}{
	Sea $(G, \circ, e)$ un grupo, $S, T \subseteq G$, $ST$, no es necesariamente un subgrupo.
}

\begin{prop}
	Sea $(G, \circ, e)$ un grupo, sean $H, K \le G$ entonces $HK$ es un subgrupo si y sólo si $HK = KH$.
\end{prop}

\begin{proof}
	\textbf{$\Rightarrow$)} Suponga que $HK \le G$, sean $hk \in HK$, entonces $h \in H$ y $k \in K$, entonces:
	$k^{-1}h^{-1} = (hk)^{-1} \in HK$, por otro lado, como $k^{-1}h^{-1} \in KH$, así: $HK = (k^{-1}h^{-1})^{-1} \in KH$, luego $HK \subseteq KH$, análogamente $KH \subseteq HK$, por lo tanto $HK = KH$.

	\textbf{$\Leftarrow$)} Supongamos que $HK = KH$, notemos que $HK \neq \emptyset$, ya que $e \circ e \in HK$.
	Sean $x, y \in HK$, entonces existe $h_1, h_2 \in H, k_1, k_2 \in K$, tal que $x = h_1 k_1$, $y = h_2 k_2$, es claro que $y^{-1} \in HK$, pues $y^{-1} = k_2^{-1}h_2^{-1} \in KH = HK$, así que probemos que $x y^{-1} \in HK$. En efecto,
	$x y^{-1} = h_1 k_1 (h_2 k_2)^{-1} = h_1 k_1 k_2^{-1}h_2^{-1}$.
	Note que $k_1 k_2^{-1}h_2^{-1} \in KH = HK$, entonces digamos $h_3 k_3 = k_1 k_2^{-1}h_2^{-1}$ con $h_3 \in H, k_3 \in K$.
	Luego:
	$h_1(k_1 k_2^{-1}h_2^{-1}) = h_1(h_3 k_3) \in HK$, por lo tanto $xy^{-1} \in HK$, y por una proposición anterior $HK$ es un subgrupo de $G$.
\end{proof}

\begin{defi}
	Sea $(G, \circ, e)$ un grupo, sean $H, K \le G$, $HK$ se llama el producto de $H$ con $K$.
\end{defi}

\CorolarioBox{}{
	Si $(G, \circ, e)$ es un grupo abeliano el producto finito de subgrupos es un subgrupo.
}

\begin{enefecto}
	es inmediato usando inducción y el hecho que el producto de subgrupos es asociativo.
\end{enefecto}


